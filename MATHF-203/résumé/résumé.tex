\documentclass{article}

\usepackage[french]{babel}
\usepackage{commath}
\usepackage{palatino, eulervm}
\usepackage{fullpage}
\usepackage[utf8]{inputenc}
\usepackage[T1]{fontenc}
\usepackage{amsthm, amsmath, amsfonts, amssymb}
\usepackage{mathtools}
%\usepackage{stmaryrd}
\usepackage[bottom]{footmisc}
\usepackage[parfill]{parskip}
\usepackage[framemethod=tikz]{mdframed}
\usepackage{hyperref}

\title{MATHF-203 -- Algèbre I}
\date{Année académique 2016 - 2017}
\author{R. Petit}

% amsthm
\newtheorem{thm}{Théorème}[section]
\newtheorem{prp}[thm]{Proposition}
\newtheorem{cor}[thm]{Corollaire}
\newtheorem{lem}[thm]{Lemme}
\addto\captionsfrench{\renewcommand\proofname{\underline{Démonstration}}}
\theoremstyle{definition}
\newtheorem{déf}[thm]{Définition}
\theoremstyle{remark}
\newtheorem*{rmq}{Remarque}
\newtheorem{ex}{Exemple}[section]

% link amsthm and mdframed
\iftrue
%\iffalse
	% pre-amsthm
	\mdfdefinestyle{resultstyle}{%
		hidealllines=true,%
		leftline=true,%
		rightline=true,%
		innerleftmargin=10pt,%
		innerrightmargin=10pt,%
		innertopmargin=10pt,%
		innerbottommargin=8pt,%
	}

	\surroundwithmdframed[style=resultstyle]{thm}
	\surroundwithmdframed[style=resultstyle]{prp}
	\surroundwithmdframed[style=resultstyle]{cor}
	\surroundwithmdframed[style=resultstyle]{lem}
\fi

\DeclareMathOperator{\Id}{Id}
\DeclareMathOperator{\SO}{SO}
\DeclareMathOperator{\ord}{ord}

\newcommand{\K}{\mathbb K}
\newcommand{\C}{\mathbb C}
\newcommand{\R}{\mathbb R}
\newcommand{\Z}{\mathbb Z}
\newcommand{\N}{\mathbb N}
\newcommand{\tq}{\text{ t.q. }}

\newcommand{\proofright}{{\framebox[1.5\width]{$\Rightarrow$}\hspace{.1cm}}}
\newcommand{\proofleft}{{\framebox[1.5\width]{$\Leftarrow$}\hspace{.1cm}}}
\newcommand{\simeqq}{\cong}

\newcommand{\eng}[1]{\left\langle#1\right\rangle}

\begin{document}
\pagenumbering{Roman}
\maketitle
\tableofcontents
\newpage
\pagenumbering{arabic}
\setcounter{page}{1}

\section{Les groupes}
	\subsection{Définitions}
		\begin{déf} Un \textit{groupe} $(G, *)$ est un ensemble non-vide $G$ muni d'une loi de composition $* : G \times G \to G$ tels que~:
		\begin{itemize}
			\item $*$ est associative~;
			\item $G$ possède un élément neutre noté $e \in G$~;
			\item chaque élément $g$ de $G$ possède un inverse noté $g^{-1}$.
		\end{itemize}
		\end{déf}

		\begin{déf} Un ensemble non-vide $M$ muni d'une loi de composition $* : M \times M \to M$ associative telle que $M$ admet un neutre par $*$ est appelé
		un \textit{monoïde}.
		\end{déf}

		\begin{déf} Un monoïde $(M, *)$ est dit \textit{abélien} (ou \textit{commutatif}) lorsque $*$ est commutative.
		\end{déf}

		\begin{rmq} Un groupe est un monoïde admettant un inverse pour chaque élément. Dès lors, les résultats et définitions sur les monoïdes s'appliquent
		également aux groupes.
		\end{rmq}

		\begin{prp} Dans un groupe $(G, *)$, les équations~:
		\begin{align}\label{eq:xa=b}
			x*a = b,
		\end{align}
		et~:
		\begin{align}\label{eq:ay=b}
			a*y = b
		\end{align}
		admettent une unique solution, i.e.~:
		\[(x, y) = (b*a^{-1}, a^{-1}*b) \in G^2.\]
		\end{prp}

		\begin{proof} $G$ est un groupe, du coup $a$ et $b$ admettent un inverse. L'existence de la solution est donc triviale.

		Soit $x$, solution de~\eqref{eq:xa=b}. On a alors~:
		\[x = x*e = x*a*a^{-1} = b*a^{-1}.\]
		Similairement pour $y$, solution de~\eqref{eq:ay=b}, on a~:
		\[y = e*y = a^{-1}*a*y = a^{-1}*b.\]
		\end{proof}

		\begin{prp} Le neutre d'un groupe est unique, et l'inverse de tout élément l'est également.

		De plus~:
		\[\forall a, b, c, d \in G : \begin{cases}&c*a = d*a \Rightarrow c=d, \\&a*c=a*d \Rightarrow c=d.\end{cases}\]
		\end{prp}

		\begin{proof} EXERCICE. %TODO
		\end{proof}

		\begin{prp} Si $G$ est un ensemble non-vide muni d'une loi de composition $*$ associative telle que~\eqref{eq:xa=b} et~\eqref{eq:ay=b} admettent une
		unique solution, alors $(G, *)$ est un groupe.
		\end{prp}

		\begin{proof} Pour chaque élément $a \in G$, prenons $e^L_a$ tel que $e^L_a*a = a$ et $e^R_a$ tel que $a*e^R_a = a$. Ces deux équations admettent une
		unique solution par hypothèse. On trouve alors~:
		\[e^L_a*a = a = a*e^R_a,\]
		d'où l'on déduit~:
		\[a*e^L_a*a = a*a=a*e^R_a*a,\]
		et donc $e^L_a = e^R_a$ en multipliant à gauche et à droite par $a^{-1}$. On en déduit l'unicité d'un neutre pour $a$ et notons-le $e_a$. Montrons que
		ce neutre l'est pour tous les éléments de $G$. Prenons $(a, b) \in G^2$ et leur neutre respectif $e_a$ et $e_b$. On peut écrire~:
		\[a*e_b*b = a*b = a*e_a*b,\]
		d'où l'on déduit $e_a = e_b$ en multipliant à gauche par $a^{-1}$ et à droite par $b^{-1}$.
		\end{proof}

		\begin{déf} Si $\abs G < \infty$, on peut définir la \textit{table de multiplication} de $(G, *)$ par un tableau de dimensions $\abs G \times \abs G$
		reprenant tous les résultats de $g*h$ pour $g, h \in G$.
		\end{déf}

	\subsection{Groupes de transformation}
		\begin{déf} Soit $S$ un ensemble non-vide. Soit $G$ l'ensemble des bijections de $S$ dans $S$. On définit la loi de composition~:
		\[\circ : G \times G \to G : (\psi, \varphi) \mapsto (\psi \circ \varphi),\]
		tels que $\forall s \in S : (\psi \circ \varphi)(s) = \psi(\varphi(s))$.
		\end{déf}

		\begin{prp} $(G, \circ)$ est un groupe (de permutation sur $S$).
		\end{prp}

		\begin{proof} Le neutre est donné par $\Id \in G$ où $\forall s \in S : \Id(s) = s$. La loi $\circ$ est trivialement associative, et l'inverse d'une
		fonction est bien définie sur les bijections.
		\end{proof}

		\begin{ex} L'ensemble $\SO(3, \R)$ des rotations axiales passant par $\mathcal O$ forme un groupe de transformations.
		\end{ex}

		\begin{rmq} Un groupe de transformation est composé de fonctions bijectives. L'ensemble $G$ est donc un ensemble fonctionnel.
		\end{rmq}

	\subsection{Sous-groupes}
		\begin{déf} Soit $(G, *)$ un groupe, et soit $S \subset G$. Si $(S, *)$ est un groupe, alors on dit que $(S, *)$ est un \textit{sous-groupe} de $(G, *)$.
		\end{déf}

		\begin{prp} Soit $(G, *)$ un groupe. $S \subseteq G$ est un sous-groupe de $G$ si et seulement si~:
		\[\forall a, b \in S : a*b^{-1} \in S.\]
		\end{prp}

		\begin{proof} \proofright Trivial car $S$ est un groupe.

		\proofleft $S$ est non-vide, donc $e \in S$ car si $a \in S$, alors par hypothèse $e = a*a^{-1} \in S$. De même, soit $a \in S$. On sait que
		$a^{-1} = e*a^{-1} \in S$. Et $S$ est stable par $*$ car si $a, b \in S$, on sait que $b^{-1} \in S$, et donc $a*\left(b^{-1}\right)^{-1} \in S$.
		\end{proof}

		\begin{prp} Si $\{S_\alpha \tq \alpha \in I\}$ est une famille de sous-groupes de $(G, *)$, alors $S \coloneqq \bigcap_{\alpha \in I}S_\alpha$ est un
		sous-groupe de $(G, *)$ également.
		\end{prp}

		\begin{proof} On sait que $e \in S$ car $e \in S_\alpha$ pour tout $\alpha \in I$. Donc $S \neq \emptyset$. Prenons $a, b \in S$. On sait que
		$a, b \in S_\alpha$ pour tout $\alpha \in I$. Donc $b^{-1}$ et $a*b^{-1}$ sont dans $S_\alpha$ pour tout $\alpha \in I$ également. Donc $a*b^{-1} \in S$.
		\end{proof}

		\begin{déf} Soit $(G, *)$ un groupe et soit $P \subseteq G$. On appelle le \textit{sous-groupe de $G$ engendré par $P$} le plus petit sous-groupe de $G$
		contenant $P$. On le note $\eng P$
		\end{déf}

		\begin{déf} Soit $(G, *)$ un groupe et soit $g \in G$. On appelle \textit{ordre de $g$} le plus petit $n \in \N^*$ tel que $g^n = e$.
		On le note $\ord(e)$.

		L'ordre de $(G, *)$ est $\abs G$.
		\end{déf}

		\begin{déf} Un groupe $(G, *)$ est dit \textit{cyclique} lorsqu'il existe $g \in G$ tel que $G = \eng g \coloneqq \eng {\{g\}}$.
		\end{déf}

	\subsection{Isomorphismes}
		\begin{déf} Un \textit{isomorphisme} entre deux groupes $(G, *)$ et $(H, \star)$ est une bijection $\phi : G \to H$ telle que~:
		\[\forall g_1, g_2 \in G : \phi(g_1 * g_2) = \phi(g_1) \star \phi(g_2).\]
		\end{déf}

		\begin{rmq} La relation «~\textit{être isomorphe}~» dans l'ensemble des groupes est une relation d'équivalence.

		De plus, sis $G$ et $H$ sont deux groupes finis, $\phi : G \to H$ est un isomorphisme si et seulement si $\phi$ est bijective, et la table de
		multiplication de $H$ par $\phi(G)$ est l'image de la table de multiplication de $G$ par $G$.
		\end{rmq}

		\begin{prp} Soit $\phi : (G, *) \to (H, \star)$ un isomorphisme de groupes. Alors~:
		\begin{itemize}
			\item $\phi(e_G) = e_H$~;
			\item $\forall g \in G : \phi(g)^1 = \phi(g^{-1})$.
		\end{itemize}
		\end{prp}

		\begin{proof} On sait que $\phi(e_G) = \phi(e_G * e_G) = \phi(e_G) \star \phi(e_G)$. Dès lors, il est évident que $\phi(e_G)$ est le neutre de $H$.

		Soit $g \in G$. On sait également que $e_H = \phi(e_G) = \phi(g * g^{-1}) = \phi(g) \star \phi(g^{-1})$. On a donc bien, en multipliant par
		$\phi(g)^{-1}$ à gauche que $\phi(g)^{-1} = \phi(g^{-1})$.
		\end{proof}

		\begin{thm} Tout groupe est isomorphe à un groupe de transformation.
		\end{thm}

		\begin{proof} Soit $(G, *)$ un groupe, et soit $g \in G$. On définit $\phi : G \to \{\ell_g \tq g \in G\}$, où $\ell_g : G \to G : h \mapsto g*h$.

		Pour $g \in G$, montrons que $\ell_g$ est bijective~:
		\begin{itemize}
			\item il est évident que $\forall g, h, h' \in G : g*h = g*h' \iff h=h'$ par les règles de simplification~;
			\item $\forall h \in G : \ell_g(g^{-1}*h) = g*g^{-1}*h = h$.
		\end{itemize}

		On a donc que $\ell_g$ est bien bijective pour tout $g \in G$.

		Montrons maintenant que $\phi : g \mapsto \ell_g$ est un isomorphisme de groupes~:
		\begin{itemize}
			\item $\phi$ est surjective par définition.
			\item soient $g, h \in G$. $\ell_g = \ell_h$ si et seulement si pour tout $\gamma \in G$, on a $g*\gamma = h*\gamma$, et donc si et seulement si
			on a $g = h$. $\phi$ est donc injective.
			\item Soient $g, h, \gamma \in G$. $\phi(g * h)(\gamma) = g*h*\gamma = g*\ell_h(\gamma) = (\ell_g \circ \ell_h)(\gamma)$.
		\end{itemize}
		\end{proof}

		\begin{thm} Tout groupe cyclique est déterminé, à isomorphisme près, par l'ordre d'un élément $g$ qui l'engendre.

		Plus précisément, si $(G, *)$ est un groupe engendré par un élément $g$ d'ordre $\ord(g)$ fini, alors $G \simeqq (\Z_{\ord(g)}, +)$~; et si $g$ est
		d'ordre infini, alors $(G, *) \simeqq (\Z, +)$.
		\end{thm}

		\begin{proof} S'il existe $g \in G$ tel que $G = \eng g$ et $\ord(g) \lneqq +\infty$, alors $G = \{e, g, \ldots, g^{\ord(g)-1}\}$.

		Soit $\phi : \Z_{\ord(g)} \to G : k \mapsto g^k$. $\phi$ est trivialement bijective, et on observe~:
		\[\phi(k+l) = g^{k+l} = g^k * g^l = \phi(k) * \phi(l).\]

		Supposons maintenant qu'il existe $g \in G$ tel que $\ord(g) = +\infty$ et $\eng g = G$. On pose~:
		\[\forall p \in \N^* : g^{-p} \coloneqq g^{-1} * g^{1-p}.\]

		Puisque $\ord(g) = +\infty$, si $g^x = g^y$ pour $x, y \in \Z$, alors $x=y$. En reprenant le même $\phi$ étendu à $\Z$, on a bien, à nouveau, un
		isomorphisme de groupes.
		\end{proof}

		\begin{cor} Tout groupe cyclique est commutatif.
		\end{cor}

		\begin{proof} Étant isomorphe à $\Z_n$ pour un certain $n \in \N$ ou à $\Z$, par passage à l'isomorphisme, la propriété d'additivité est conservée.
		\end{proof}

		\begin{prp} Si $(G, *)$ est un groupe cyclique, tout sous-groupe $S$ de $G$ est cyclique.
		\end{prp}

		\begin{proof} Prenons $a \in G$ tel que $G = \eng a$. Posons $N \coloneqq \{n \in \Z \tq a^n \in S\}$. Dans le cas fini, prenons $\underline n$, le
		plus petit entier positif de $N$. Supposons par l'absurde que $a^t \in S$ ne soit pas une puissance entière de $a^{\underline n}$. Par Euclide, on a~:
		\[\exists (q, r) \in \Z \times \N \tq t = q\underline n + r,\]
		et donc~:
		\[a^t = a^{q\underline n} * a^r,\]
		avec $0 \lneqq r \lneqq \underline n$. On sait que $a^t \in S$, et $a^{q\underline n} \in S$ (donc $a^{-q\underline n} \in S$). Dès lors, $a^r \in S$.
		Or $\underline n$ est le plus petit entier positif tel que $a^{\underline n} \in S$. Il y a donc contradiction, et $a^t$ est une puissance entière de
		$a^{\underline n}$. Dès lors, $S = \eng {a^{\underline n}}$.
		\end{proof}

	\subsection{Classes latérales et théorème de Lagrange}

\end{document}
