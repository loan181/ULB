\documentclass{article}

\usepackage[french]{babel}
\usepackage{commath}
\usepackage{palatino, eulervm}
\usepackage{fullpage}
\usepackage[utf8]{inputenc}
\usepackage[T1]{fontenc}
\usepackage{amsthm, amsmath, amsfonts, amssymb}
\usepackage{mathtools}
%\usepackage{stmaryrd}
\usepackage[bottom]{footmisc}
\usepackage[parfill]{parskip}
\usepackage[framemethod=tikz]{mdframed}
\usepackage{hyperref}

\title{MATHF-203 -- Algèbre I}
\date{Année académique 2016 - 2017}
\author{R. Petit}

% amsthm
\newtheorem{thm}{Théorème}[section]
\newtheorem{prp}[thm]{Proposition}
\newtheorem{cor}[thm]{Corollaire}
\newtheorem{lem}[thm]{Lemme}
\addto\captionsfrench{\renewcommand\proofname{\underline{Démonstration}}}
\theoremstyle{definition}
\newtheorem{déf}[thm]{Définition}
\theoremstyle{remark}
\newtheorem*{rmq}{Remarque}
\newtheorem{ex}{Exemple}[section]

% link amsthm and mdframed
\iftrue
%\iffalse
	% pre-amsthm
	\mdfdefinestyle{resultstyle}{%
		hidealllines=true,%
		leftline=true,%
		rightline=true,%
		innerleftmargin=10pt,%
		innerrightmargin=10pt,%
		innertopmargin=10pt,%
		innerbottommargin=8pt,%
	}

	\surroundwithmdframed[style=resultstyle]{thm}
	\surroundwithmdframed[style=resultstyle]{prp}
	\surroundwithmdframed[style=resultstyle]{cor}
	\surroundwithmdframed[style=resultstyle]{lem}
\fi

\DeclareMathOperator{\Id}{Id}
\DeclareMathOperator{\SO}{SO}
\DeclareMathOperator{\ord}{ord}
\DeclareMathOperator{\Imf}{Im}
\DeclareMathOperator{\Ker}{Ker}
\DeclareMathOperator{\Aut}{Aut}

\newcommand{\K}{\mathbb K}
\newcommand{\C}{\mathbb C}
\newcommand{\R}{\mathbb R}
\newcommand{\Z}{\mathbb Z}
\newcommand{\N}{\mathbb N}
\newcommand{\tq}{\text{ t.q. }}

\newcommand{\proofright}{{\framebox[1.5\width]{$\Rightarrow$}\hspace{.1cm}}}
\newcommand{\proofleft}{{\framebox[1.5\width]{$\Leftarrow$}\hspace{.1cm}}}
\newcommand{\simeqq}{\cong}

\newcommand{\eng}[1]{\left\langle#1\right\rangle}

\begin{document}
\pagenumbering{Roman}
\maketitle
\tableofcontents
\newpage
\pagenumbering{arabic}
\setcounter{page}{1}

\section{Les groupes}
	\subsection{Définitions}
		\begin{déf} Un \textit{groupe} $(G, *)$ est un ensemble non-vide $G$ muni d'une loi de composition $* : G \times G \to G$ tels que~:
		\begin{itemize}
			\item $*$ est associative~;
			\item $G$ possède un élément neutre noté $e \in G$~;
			\item chaque élément $g$ de $G$ possède un inverse noté $g^{-1}$.
		\end{itemize}
		\end{déf}

		\begin{déf} Un ensemble non-vide $M$ muni d'une loi de composition $* : M \times M \to M$ associative telle que $M$ admet un neutre par $*$ est appelé
		un \textit{monoïde}.
		\end{déf}

		\begin{déf} Un monoïde $(M, *)$ est dit \textit{abélien} (ou \textit{commutatif}) lorsque $*$ est commutative.
		\end{déf}

		\begin{rmq} Un groupe est un monoïde admettant un inverse pour chaque élément. Dès lors, les résultats et définitions sur les monoïdes s'appliquent
		également aux groupes.
		\end{rmq}

		\begin{prp} Dans un groupe $(G, *)$, les équations~:
		\begin{align}\label{eq:xa=b}
			x*a = b,
		\end{align}
		et~:
		\begin{align}\label{eq:ay=b}
			a*y = b
		\end{align}
		admettent une unique solution, i.e.~:
		\[(x, y) = (b*a^{-1}, a^{-1}*b) \in G^2.\]
		\end{prp}

		\begin{proof} $G$ est un groupe, du coup $a$ et $b$ admettent un inverse. L'existence de la solution est donc triviale.

		Soit $x$, solution de~\eqref{eq:xa=b}. On a alors~:
		\[x = x*e = x*a*a^{-1} = b*a^{-1}.\]
		Similairement pour $y$, solution de~\eqref{eq:ay=b}, on a~:
		\[y = e*y = a^{-1}*a*y = a^{-1}*b.\]
		\end{proof}

		\begin{prp} Le neutre d'un groupe est unique, et l'inverse de tout élément l'est également.

		De plus~:
		\[\forall a, b, c, d \in G : \begin{cases}&c*a = d*a \Rightarrow c=d, \\&a*c=a*d \Rightarrow c=d.\end{cases}\]
		\end{prp}

		\begin{proof} EXERCICE. %TODO
		\end{proof}

		\begin{prp} Si $G$ est un ensemble non-vide muni d'une loi de composition $*$ associative telle que~\eqref{eq:xa=b} et~\eqref{eq:ay=b} admettent une
		unique solution, alors $(G, *)$ est un groupe.
		\end{prp}

		\begin{proof} Pour chaque élément $a \in G$, prenons $e^L_a$ tel que $e^L_a*a = a$ et $e^R_a$ tel que $a*e^R_a = a$. Ces deux équations admettent une
		unique solution par hypothèse. On trouve alors~:
		\[e^L_a*a = a = a*e^R_a,\]
		d'où l'on déduit~:
		\[a*e^L_a*a = a*a=a*e^R_a*a,\]
		et donc $e^L_a = e^R_a$ en multipliant à gauche et à droite par $a^{-1}$. On en déduit l'unicité d'un neutre pour $a$ et notons-le $e_a$. Montrons que
		ce neutre l'est pour tous les éléments de $G$. Prenons $(a, b) \in G^2$ et leur neutre respectif $e_a$ et $e_b$. On peut écrire~:
		\[a*e_b*b = a*b = a*e_a*b,\]
		d'où l'on déduit $e_a = e_b$ en multipliant à gauche par $a^{-1}$ et à droite par $b^{-1}$.
		\end{proof}

		\begin{déf} Si $\abs G < \infty$, on peut définir la \textit{table de multiplication} de $(G, *)$ par un tableau de dimensions $\abs G \times \abs G$
		reprenant tous les résultats de $g*h$ pour $g, h \in G$.
		\end{déf}

	\subsection{Groupes de transformation}
		\begin{déf} Soit $S$ un ensemble non-vide. Soit $G$ l'ensemble des bijections de $S$ dans $S$. On définit la loi de composition~:
		\[\circ : G \times G \to G : (\psi, \varphi) \mapsto (\psi \circ \varphi),\]
		tels que $\forall s \in S : (\psi \circ \varphi)(s) = \psi(\varphi(s))$.
		\end{déf}

		\begin{prp} $(G, \circ)$ est un groupe (de permutation sur $S$).
		\end{prp}

		\begin{proof} Le neutre est donné par $\Id \in G$ où $\forall s \in S : \Id(s) = s$. La loi $\circ$ est trivialement associative, et l'inverse d'une
		fonction est bien définie sur les bijections.
		\end{proof}

		\begin{ex} L'ensemble $\SO(3, \R)$ des rotations axiales passant par $\mathcal O$ forme un groupe de transformations.
		\end{ex}

		\begin{rmq} Un groupe de transformation est composé de fonctions bijectives. L'ensemble $G$ est donc un ensemble fonctionnel.
		\end{rmq}

	\subsection{Sous-groupes}
		\begin{déf} Soit $(G, *)$ un groupe, et soit $S \subset G$. Si $(S, *)$ est un groupe, alors on dit que $(S, *)$ est un \textit{sous-groupe} de $(G, *)$.
		\end{déf}

		\begin{prp} Soit $(G, *)$ un groupe. $S \subseteq G$ est un sous-groupe de $G$ si et seulement si~:
		\[\forall a, b \in S : a*b^{-1} \in S.\]
		\end{prp}

		\begin{proof} \proofright Trivial car $S$ est un groupe.

		\proofleft $S$ est non-vide, donc $e \in S$ car si $a \in S$, alors par hypothèse $e = a*a^{-1} \in S$. De même, soit $a \in S$. On sait que
		$a^{-1} = e*a^{-1} \in S$. Et $S$ est stable par $*$ car si $a, b \in S$, on sait que $b^{-1} \in S$, et donc $a*\left(b^{-1}\right)^{-1} \in S$.
		\end{proof}

		\begin{prp} Si $\{S_\alpha \tq \alpha \in I\}$ est une famille de sous-groupes de $(G, *)$, alors $S \coloneqq \bigcap_{\alpha \in I}S_\alpha$ est un
		sous-groupe de $(G, *)$ également.
		\end{prp}

		\begin{proof} On sait que $e \in S$ car $e \in S_\alpha$ pour tout $\alpha \in I$. Donc $S \neq \emptyset$. Prenons $a, b \in S$. On sait que
		$a, b \in S_\alpha$ pour tout $\alpha \in I$. Donc $b^{-1}$ et $a*b^{-1}$ sont dans $S_\alpha$ pour tout $\alpha \in I$ également. Donc $a*b^{-1} \in S$.
		\end{proof}

		\begin{déf} Soit $(G, *)$ un groupe et soit $P \subseteq G$. On appelle le \textit{sous-groupe de $G$ engendré par $P$} le plus petit sous-groupe de $G$
		contenant $P$. On le note $\eng P$
		\end{déf}

		\begin{déf} Soit $(G, *)$ un groupe et soit $g \in G$. On appelle \textit{ordre de $g$} le plus petit $n \in \N^*$ tel que $g^n = e$.
		On le note $\ord(e)$.

		L'ordre de $(G, *)$ est $\abs G$.
		\end{déf}

		\begin{déf} Un groupe $(G, *)$ est dit \textit{cyclique} lorsqu'il existe $g \in G$ tel que $G = \eng g \coloneqq \eng {\{g\}}$.
		\end{déf}

	\subsection{Isomorphismes}
		\begin{déf} Un \textit{isomorphisme} entre deux groupes $(G, *)$ et $(H, \star)$ est une bijection $\phi : G \to H$ telle que~:
		\[\forall g_1, g_2 \in G : \phi(g_1 * g_2) = \phi(g_1) \star \phi(g_2).\]
		\end{déf}

		\begin{rmq} La relation «~\textit{être isomorphe}~» dans l'ensemble des groupes est une relation d'équivalence.

		De plus, sis $G$ et $H$ sont deux groupes finis, $\phi : G \to H$ est un isomorphisme si et seulement si $\phi$ est bijective, et la table de
		multiplication de $H$ par $\phi(G)$ est l'image de la table de multiplication de $G$ par $G$.
		\end{rmq}

		\begin{prp}\label{prp:prop_isomorphisme} Soit $\phi : (G, *) \to (H, \star)$ un isomorphisme de groupes. Alors~:
		\begin{itemize}
			\item $\phi(e_G) = e_H$~;
			\item $\forall g \in G : \phi(g)^1 = \phi(g^{-1})$.
		\end{itemize}
		\end{prp}

		\begin{proof} On sait que $\phi(e_G) = \phi(e_G * e_G) = \phi(e_G) \star \phi(e_G)$. Dès lors, il est évident que $\phi(e_G)$ est le neutre de $H$.

		Soit $g \in G$. On sait également que $e_H = \phi(e_G) = \phi(g * g^{-1}) = \phi(g) \star \phi(g^{-1})$. On a donc bien, en multipliant par
		$\phi(g)^{-1}$ à gauche que $\phi(g)^{-1} = \phi(g^{-1})$.
		\end{proof}

		\begin{thm} Tout groupe est isomorphe à un groupe de transformation.
		\end{thm}

		\begin{proof} Soit $(G, *)$ un groupe, et soit $g \in G$. On définit $\phi : G \to \{\ell_g \tq g \in G\}$, où $\ell_g : G \to G : h \mapsto g*h$.

		Pour $g \in G$, montrons que $\ell_g$ est bijective~:
		\begin{itemize}
			\item il est évident que $\forall g, h, h' \in G : g*h = g*h' \iff h=h'$ par les règles de simplification~;
			\item $\forall h \in G : \ell_g(g^{-1}*h) = g*g^{-1}*h = h$.
		\end{itemize}

		On a donc que $\ell_g$ est bien bijective pour tout $g \in G$.

		Montrons maintenant que $\phi : g \mapsto \ell_g$ est un isomorphisme de groupes~:
		\begin{itemize}
			\item $\phi$ est surjective par définition.
			\item soient $g, h \in G$. $\ell_g = \ell_h$ si et seulement si pour tout $\gamma \in G$, on a $g*\gamma = h*\gamma$, et donc si et seulement si
			on a $g = h$. $\phi$ est donc injective.
			\item Soient $g, h, \gamma \in G$. $\phi(g * h)(\gamma) = g*h*\gamma = g*\ell_h(\gamma) = (\ell_g \circ \ell_h)(\gamma)$.
		\end{itemize}
		\end{proof}

		\begin{thm} Tout groupe cyclique est déterminé, à isomorphisme près, par l'ordre d'un élément $g$ qui l'engendre.

		Plus précisément, si $(G, *)$ est un groupe engendré par un élément $g$ d'ordre $\ord(g)$ fini, alors $G \simeqq (\Z_{\ord(g)}, +)$~; et si $g$ est
		d'ordre infini, alors $(G, *) \simeqq (\Z, +)$.
		\end{thm}

		\begin{proof} S'il existe $g \in G$ tel que $G = \eng g$ et $\ord(g) \lneqq +\infty$, alors $G = \{e, g, \ldots, g^{\ord(g)-1}\}$.

		Soit $\phi : \Z_{\ord(g)} \to G : k \mapsto g^k$. $\phi$ est trivialement bijective, et on observe~:
		\[\phi(k+l) = g^{k+l} = g^k * g^l = \phi(k) * \phi(l).\]

		Supposons maintenant qu'il existe $g \in G$ tel que $\ord(g) = +\infty$ et $\eng g = G$. On pose~:
		\[\forall p \in \N^* : g^{-p} \coloneqq g^{-1} * g^{1-p}.\]

		Puisque $\ord(g) = +\infty$, si $g^x = g^y$ pour $x, y \in \Z$, alors $x=y$. En reprenant le même $\phi$ étendu à $\Z$, on a bien, à nouveau, un
		isomorphisme de groupes.
		\end{proof}

		\begin{cor} Tout groupe cyclique est commutatif.
		\end{cor}

		\begin{proof} Étant isomorphe à $\Z_n$ pour un certain $n \in \N$ ou à $\Z$, par passage à l'isomorphisme, la propriété d'additivité est conservée.
		\end{proof}

		\begin{prp} Si $(G, *)$ est un groupe cyclique, tout sous-groupe $S$ de $G$ est cyclique.
		\end{prp}

		\begin{proof} Prenons $a \in G$ tel que $G = \eng a$. Posons $N \coloneqq \{n \in \Z \tq a^n \in S\}$. Dans le cas fini, prenons $\underline n$, le
		plus petit entier positif de $N$. Supposons par l'absurde que $a^t \in S$ ne soit pas une puissance entière de $a^{\underline n}$. Par Euclide, on a~:
		\[\exists (q, r) \in \Z \times \N \tq t = q\underline n + r,\]
		et donc~:
		\[a^t = a^{q\underline n} * a^r,\]
		avec $0 \lneqq r \lneqq \underline n$. On sait que $a^t \in S$, et $a^{q\underline n} \in S$ (donc $a^{-q\underline n} \in S$). Dès lors, $a^r \in S$.
		Or $\underline n$ est le plus petit entier positif tel que $a^{\underline n} \in S$. Il y a donc contradiction, et $a^t$ est une puissance entière de
		$a^{\underline n}$. Dès lors, $S = \eng {a^{\underline n}}$.
		\end{proof}

	\subsection{Classes latérales et théorème de Lagrange}
		Dans cette sous-section, considérons que $(G, *)$ est un groupe, et que $S$ est un sous-groupe de $G$.

		\begin{déf} Soient $(G, *)$ un groupe, et $S$ un sous-groupe de $G$. On appelle \textit{classe latérale gauche de $S$ par $a \in G$} l'ensemble~:
		\[a*S \coloneqq \left\{a*s \tq s \in S\right\}.\]

		Similairement, une classe latérale droite est sous la forme~:
		\[S*a \coloneqq \left\{s*a \tq s \in S\right\},\]
		pour un certain $a \in G$.
		\end{déf}

		\begin{prp} Deux classes latérales gauches $a*S$ et $b*S$ sont identiques ou sont d'intersection nulle.
		\end{prp}

		\begin{proof} Soient deux telles classes d'équivalences, telles que $a*S \cap b*S \neq \emptyset$. On sait alors qu'il existe $c \in a*S \cap b*S$.
		On en déduit $u, v \in S$ tels que $a*u = c = b*v$. On a alors, pour $s \in S$~:
		\[b*s = c*v^{-1}*s = a*u*v^{-1}*s.\]
		Or $u$ et $v^{-1}$ sont dans $S$. Donc $b*S \subseteq a*S$.

		Par un raisonnement similaire, on a $a*S \subseteq b*S$, et donc $a*S = b*S$.
		\end{proof}

		\begin{prp}\label{prp:prop_classes_lat} Deux éléments $a, b \in G$ ont la même classe latérale gauche, i.e.  $a*S = b*S$ si et seulement si
		$a^{-1}*b \in S$.
		\end{prp}

		\begin{proof} \proofright $a*S = b*S = a*a^{-1}*b*S$. Donc $a^{-1}*b \in S$.

		\proofleft On sait $S = e*S = s*S$ pour tout $s \in S$. Or $a^{-1}*b \in S$. Donc $S = a^{-1}*b*S$, ou encore $a*S = b*S$.
		\end{proof}

		\begin{déf} On appelle \textit{indice} d'un sous-groupe $S$ de $G$ le «~nombre~» de classes latérales gauches distinctes de $S$ dans $G$.
		\end{déf}

		\begin{prp} Il existe une bijection entre un sous-groupe $S$ et chacune de ses classes latérales.
		\end{prp}

		\begin{proof} Soit $a \in G$. On considère $\phi : S \to a*S : s \mapsto a*s$.

		$\phi$ est surjective par définition, et est trivialement injective.
		\end{proof}

		\begin{cor} Il existe une bijection entre tout groupe $(G, *)$ et l'ensemble $S \times T$, pour $S$, un sous-groupe de $G$ et $T$ l'ensemble des classes
		latérales gauches distinctes de $S$ dans $G$.
		\end{cor}

		\begin{thm}[Théorème de Lagrange] Soient $(G, *)$ un groupe fini, et $S$ un sous-groupe de $G$. L'ordre de $G$ est un multiple de l'ordre de $S$.
		\end{thm}

		\begin{proof} Trivial par le corollaire précédent.
		\end{proof}

		\begin{cor} Si $p$ est un nombre premier, tout groupe $(G, *)$ à $p$ éléments est isomorphe à $(\Z_p, +)$.
		\end{cor}

		\begin{proof} $p$ est premier, donc $\abs G \geq 2$. Soit $g$ tel que $\ord(g) > 1$ (c-à-d $g \neq e$). $S \coloneqq \eng g$ est un sous-groupe de $G$,
		on en déduit par Lagrange que $\abs {\eng g}$ divise $\abs G$. Donc $\abs S \in \{1, p\}$. Et comme $e, g \in S$ pour $g \neq e$, on a $\abs S \geq 2$,
		et donc $\abs S = p$.
		\end{proof}

		\begin{cor} Si $G$ est un groupe d'ordre fini $n$, pour tout $g \in G$, on a $\ord(g)$ divise $n$.
		\end{cor}

		\begin{cor} Il n'existe, à isomorphisme près, que deux groupes d'ordre 4, à savoir $\Z_4$ et $V_4 \coloneqq \Z_2 \times \Z_2$.
		\end{cor}

		\begin{cor} Tout groupe d'ordre fini $n  \leq 5$ est commutatif.
		\end{cor}

	\subsection{Sous-groupes normaux et homomorphismes}
		\begin{déf} Un sous-groupe $N$ d'un groupe $(G, *)$ est dit \textit{normal} lorsque ses classes latérales gauches sont des classes latérales droites, i.e.~:
		\[\forall a \in G : a*N = N*a.\]
		\end{déf}

		\begin{prp} Soit $N$, un sous-groupe d'un groupe $(G, *)$. Les assertions suivantes sont équivalentes~:
		\begin{enumerate}
			\item $N$ est normal~;
			\item $\forall a \in G : a*N = N*a$~;
			\item $\forall a \in G : a*N \subseteq N*a$~;
			\item $\forall a \in G : a*N*a^{-1} \subseteq N$.
		\end{enumerate}
		\end{prp}

		\begin{proof} On sait, par définition, que $1 \iff 2$, et $2 \Rightarrow 3$. En multipliant par $a^{-1}$ à droite, on a $3 \iff 4$.

		Montrons donc que $3 \Rightarrow 2$. En particulier, c'est vrai pour $a^{-1}$, et donc $a^{-1}*N \subseteq N*a^{-1}$, ou encore $N*a \subseteq a*N$.
		Avec $3$, cela implique que $a*N = N*a$.
		\end{proof}

		\begin{déf} Un \textit{homomorphisme} d'un groupe $(G, *)$ dans un groupe $(H, \star)$ est une application $f : G \to H$ telle que~:
		\[\forall g_1, g_2 \in G : f(g_1 * g_2) = f(g_1) \star f(g_2).\]
		\end{déf}

		\begin{prp} Si $f$ est un homomorphisme de $(G, *)$ dans $(H, \star)$, alors l'image du neutre par $f$ est le neutre de $H$, et l'inverse $g^{-1}$ d'un
		élément $g \in G$ est envoyé sur $f(g)^{-1} \in H$.
		\end{prp}

		\begin{proof} Voir preuve de la Proposition~\ref{prp:prop_isomorphisme}.
		\end{proof}

		\begin{prp} Si $f$ est un homomorphisme de $(G, *)$ dans $(H, \star)$, alors~:
		\begin{enumerate}
			\item $\Imf f \coloneqq \{f(g) \tq g \in G\} \eqqcolon f(G)$ est un sous-groupe de $H$~;
			\item $\Ker f \coloneqq \{g \in G \tq f(g) = e_H\}$ est un sous-groupe \textbf{normal} de $H$.
		\end{enumerate}
		\end{prp}

		\begin{proof} Montrons d'abord que $\Imf f \leq H$. Prenons donc $g, g' \in G$. On a alors $f(g) \star f(g')^{-1} = f(g*{g'}^{-1}) \in \Imf f$ car $f$
		est un homomorphisme.

		Montrons ensuite que $\Ker f \leq G$. Prenons $g_1, g_2 \in \Ker f$. On sait donc que~:
		\[f(g_1 * g_2^{-1}) = f(g_1) \star f(g_2)^{-1} = e_H \star e_H^{-1} = e_H.\]
		On en déduit que $g_1*g_2^{-1} \in \Ker f$.

		Montrons alors que pour tout $a \in G$, on a $a*\Ker(f)*a^{-1} \subseteq \Ker f$. Soit $g \in \Ker f$. On calcule~:
		\[f\left(a*g*a^{-1}\right) = f(a)*e_H*f(a)^{-1} = f(a)*f(a)^{-1} = e_H,\]
		et donc $a*g*a^{-1} \in \Ker f$, ou encore $a*\Ker(f)*a^{-1} \subseteq \Ker f$. $\Ker f$ est donc bien un sous-groupe normal de $G$.
		\end{proof}

		\begin{rmq} Un isomorphisme est un homomorphisme tel que $\Ker f = \{e_G\}$ et $\Imf f = H$ car $f$ est respectivement injective et surjective.
		\end{rmq}

		\begin{déf} Un homomorphisme d'un groupe $(G, *)$ dans lui-même est appelé un \textit{automorphisme}.
		\end{déf}

		\begin{déf} Pour tout $g \in G$, on définit la \textit{conjugaison par $g$} comme la fonction~:
		\[c(g) : G \to G : h \mapsto g*h*g^{-1}.\]
		\end{déf}

		\begin{prp} Pour $g \in G$, la conjugaison par $g$ est un automorphisme.
		\end{prp}

		\begin{proof} $c(g)$ est injective par les règles de simplification, et est surjective car pour tout $\gamma \in G$, on a~:
		\[c(g)^{-1}(\gamma) = g^{-1}*\gamma*g,\]
		en effet~:
		\[c(g)(g^{-1}*\gamma*g) = g*\left(g^{-1}*\gamma*g\right)*g^{-1} = \gamma.\]
		Donc $c(g)$ est bien surjective.

		Et puisque $c(g)$ va de $G$ dans $G$, c'est bien un automorphisme.
		\end{proof}

		\begin{prp} L'application $C : G \to \Aut(G) \subset G^G : g \mapsto c(g)$ est un homomorphisme.
		\end{prp}

		\begin{proof} Soient $g, h, \gamma \in G$. On calcule~:
		\begin{align*}
			C(g*h)(\gamma) &= c(g*h)(\gamma) = (g*h)*\gamma*(g*h)^{-1} = g*h*\gamma*h^{-1}*g^{-1} = c(g)(h*\gamma*h^{-1}) \\
			&= c(g)\left(c(h)(\gamma)\right) = \left(c(g) \circ c(h)\right)(\gamma).
		\end{align*}
		\end{proof}

		\begin{prp} Soit $f$ un homomorphisme de groupe de $(G, *)$ dans $(H, \star)$. Deux éléments $x, y \in G$ ont la même image par $f$ si et seulement si
		ils appartiennent à la même classe latérale de $\Ker f$ dans $G$.
		\end{prp}

		\begin{proof} Soient $x$ et $y$ tels que $f(x) = f(y)$. On calcule~:
		\[f(x*y^{-1}) = f(x) \star f(y^{-1}) = f(y) \star f(y)^{-1} = e_H.\]
		Donc $x*y^{-1} \in \Ker f$, et donc $x * \Ker f = y * \Ker f$ par la Proposition~\ref{prp:prop_classes_lat}.
		\end{proof}

	\subsection{Groupes quotients}
		Soit $f : (G, *) \to (H, \star)$ un homomorphisme surjectif. Si $x' \in H$, alors il existe $x \in G$ tel que $f(x) = x'$. On a alors~:
		\[f^{-1}(\{x'\}) = x*\Ker f.\]
		Si $\abs {\Ker f} \geq 2$, prenons également $y' \in H$ et donc $y \in G$ tel que $f^{-1}(\{y'\}) = y*\Ker f$.

		On peut ensuite calculer~:
		\[x' \star y' = f(x) \star f(y) = f(x*y),\]
		d'où l'on déduit~:
		\[f^{-1}(\{x'*y'\}) = x*y*\Ker f.\]

		\begin{déf} Soit $N$ un sous-groupe normal du groupe $(G, *)$. On désigne par $G/N$ l'ensemble des classes latérales de $N$ dans $G$. Cela se lit
		\textit{$G$ quotienté $N$}.
		\end{déf}

		\begin{prp} Soient $x*N$ et $y*N$ deux éléments de $G/N$. Si $x' \in x*N$ et $y' \in y*N$, alors~:
		\[x'*y' \in x*y*N.\]
		\end{prp}

		\begin{proof} On sait qu'il existe $m, n \in N$ tels que $x' = x*n$ et $y' = y*m$. Par normalité de $N$, on sait qu'il existe $m' \in N$ tel que
		$y' = n*y = y*n'$. Dès lors~:
		\[x'*y' = x*n*y*n'*m.\]
		Or, $n'*m \in N$. Donc~:
		\[x'*y' \in x*y*N.\]
		\end{proof}

		\begin{déf} On définit le produit $\bar * : G/N \times G/N \to G/N : (x*N, y*N) \mapsto (x*N) \bar * (y*N) \coloneqq x*y*N$.
		\end{déf}

		\begin{thm} Soient $(G, *)$ un groupe et $N$ un sous-groupe normal de $G$. Alors $(G/N, \bar *)$ est un groupe.
		\end{thm}

		\begin{proof} $\bar *$ est interne par définition et par la proposition précédente, et est associative par associativité de $*$.

		$(G/N, \bar *)$ admet pour neutre $e*N = N$.

		Soit $g*N \in G/N$. Il admet pour inverse $g^{-1}*N$ car $(g*N) \bar * (g^{-1}*N) = e*N = N$.
		\end{proof}

		\begin{déf} Le groupe $(G/N, \bar *)$ est appelé le \textit{groupe quotient de $G$ par $N$}.
		\end{déf}

		\begin{déf} Soient $(G, *)$ un groupe, et $N \leq G$. La projection $\pi_N : G \to G/N : g \mapsto g*N$ est appelée la \textit{projection canonique}.
		\end{déf}

		\begin{prp} $\pi_N : (G, *) \to (G/N, \bar *)$ est un homomorphisme.
		\end{prp}

		\begin{proof} Soient $g, h \in G$. $\pi_N(g*h) = g*h*N = (g*N) \bar * (h*N) = \pi_N(g) \bar * \pi_N(h)$.
		\end{proof}

		\begin{prp} Si $(G, *)$ et $(H, \star)$ sont deux groupes, $f : G \to H$ est un homomorphisme, et si $N$ est un sous-groupe normal de $G$ contenu dans
		$\Ker f$, alors il existe un homomorphisme $\overline f : G/N \to H$ avec $\overline f(g*N) = f(g)$. De plus~:
		\[\Imf \overline f = \Imf f \qquad\qquad \text{ et } \qquad\qquad \Ker \overline f = \Ker(f)/N.\]
		\end{prp}

		\begin{proof} $\overline f$ est bien définie. Soient $g, g' \in G$. On calcule~:
		\[\overline f((g*N) \bar * (g'*N)) = \overline f(g*g'*N) = f(g*g').\]
		Par propriétés de morphismes de $f$, on trouve donc~:
		\[\overline g((g*N)\bar *(g'*N)) = f(g) \star f(g') = \overline f(g*N) \star \overline f(g'*N).\]

		$\Imf f = \Imf \overline f$ de manière triviale.

		$g*N \in \Ker \overline f$ si et seulement si $g \in \Ker f$. On a alors bien $\Ker \overline f = \{h*N \tq h \in \Ker f\} = \Ker(f)/N$.
		\end{proof}

	\subsection{Théorèmes d'isomorphisme}
		\subsubsection{Premier théorème d'isomorphisme}
			\begin{thm} Si $f : (G, *) \to (H, \star)$ est un homomorphisme de groupes, il induit un isomorphisme~:
			\[G/\Ker(f) \simeqq \Imf f.\]
			\end{thm}

			\begin{proof} $(\Imf f, \star)$ est un groupe car $\Imf f$ est un sous-groupe de $H$. $f : G \to \Imf f$ est un homomorphisme.

			Considérons $N = \Ker f$. $\overline f : G/\Ker(f) \to \Imf f$ est surjectif car $\Imf f = \Imf \overline f$ et est injectif car
			$\Ker(\overline f) = \Ker(f)/\Ker(f) = \{e * \Ker f\} = \{e\}$. $\overline f$ est donc un homomorphisme bijectif, ou encore, un isomorphisme.
			\end{proof}

		\subsubsection{Deuxième théorème d'isomorphisme}
			\begin{thm} Si $K, N$ sont des sous-groupes de $(G, *)$, alors $K/(N \cap K) \simeqq (N*K)/N$, où on définit~:
			\[N*K \coloneqq \left\{n*k \tq (n, k) \in N \times K\right\}.\]
			\end{thm}

			\begin{proof} Montrons que $N*K$ est un sous-groupe de $G$, i.e. $\forall x, y \in N*K : x*y^{-1} \in N*K$. Prenons donc $(x, y) \in (N*K)^2$.
			Il existe $n_x, n_y \in N$ et $k_x, k_y \in K$ tels que $(x, y) = (n_x*k_x, n_y*k_y)$. On a alors~:
			\[x*y^{-1} = n_x*k_x*k_y^{-1}*n_y^{-1}.\]
			En posant $k \coloneqq k_x*k_y^{-1} \in K$ (car $K$ est un sous-groupe), on trouve~:
			\[x*y^{-1} = n_x*k*n_y.\]

			Par normalité de $N$ on sait qu'il existe $n \in N$ tel que $k*n_y = n*k$. On trouve alors~:
			\[x*y^{-1} = n_x*n*k = (n_x*n)*k \in N*K.\]

			On observe maintenant que si $N$ est normal dans $G$, alors il l'est dans $N*K$.

			L'application $f : K \to (N*K)/N : k \mapsto k*N$ est un homomorphisme. Son noyau est $\Ker f = \{k \in K \tq k*N = N\} = K \cap N$.
			$K \cap N$ est donc un sous-groupe normal de $K$. Par le théorème précédent, on a~:
			\[K/\Ker(f) \simeqq \Imf f = (N*K)/N.\]
			\end{proof}

		\subsubsection{Troisième théorème d'isomorphisme}
			\begin{thm} Si $K$ et $N$ sont deux sous-groupes normaux de $(G, *)$ tels que $K \subseteq N$, alors $N/K$ est normal dans $G/K$ et~:
			\[\left(G/K\right)/\left(N/K\right) \simeqq G/N.\]
			\end{thm}

			\begin{proof} Prenons $\pi_N : G \to G/N$ l'homomorphisme canonique. On a $K \subset \Ker \pi = N$. Par le théorème précédent, il existe un
			homomorphisme $\overline \pi : G/K \to G/N : g*K \mapsto g*N$ surjectif de noyau $\Ker \overline \pi = N/K$.

			Puisque l'on en déduit $G/K$ normal dans $N/K$, en appliquant le premier théorème d'isomorphisme à $\overline \pi$, on trouve~:
			\[\left(G/K\right)/\left(N/K\right) \simeqq G/N.\]
			\end{proof}

\end{document}
