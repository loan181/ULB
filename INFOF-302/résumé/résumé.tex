\documentclass{article}

\usepackage[french]{babel}
\usepackage{commath}
\usepackage[utf8]{inputenc}
\usepackage[T1]{fontenc}
\usepackage{palatino}
\usepackage{eulervm}
\usepackage{fullpage}
\usepackage[parfill]{parskip}

\usepackage{amsmath}
\usepackage{amssymb}
\usepackage{amsfonts}
\usepackage{amsthm}
\usepackage{stmaryrd}
\usepackage{mathtools}

% amsthm
\newtheorem{thm}{Théorème}[section]
\newtheorem{prp}[thm]{Proposition}
\newtheorem{cor}[thm]{Corollaire}
\newtheorem{lem}[thm]{Lemme}
\newtheorem{axiome}[thm]{Axiome}
\addto\captionsfrench{\renewcommand\proofname{\underline{Démonstration}}}
\theoremstyle{definition}
\newtheorem{déf}[thm]{Définition}
\theoremstyle{remark}
\newtheorem*{rmq}{Remarque}
\newtheorem{ex}{Exemple}[section]

\title{INFOF-302 --- Informatique fondamentale}
\author{R. Petit}
\date{Année académique 2016 - 2017}

\newcommand{\intint}[2]{\left\llbracket#1, #2\right\rrbracket}
\newcommand{\logeval}[1]{\left\llbracket#1\right\rrbracket}
\newcommand{\N}{\mathbb N}
\newcommand{\R}{\mathbb R}

\begin{document}
\pagenumbering{Roman}
\maketitle
\tableofcontents
\newpage
\pagenumbering{arabic}
\setcounter{page}{1}

\section{Introduction et Logique}
	\begin{déf} \textit{Réduire} un problème de décision $A$ en un problème de décision $B$ correspond à trouver un algorithme permettant d'encoder toute
	entrée $I_A$ du problème $A$ en une entrée $I_B$ du problème $B$ telle que $I_A$ a une solution pour le problème $A$ si et seulement si $I_B$
	a une solution pour le problème $B$.
	\end{déf}

	\begin{déf} \textit{Réduire} un problème général $A$ en un problème $B$ correspond à trouver un algorithme d'encodage de toute entrée $I_A$ du problème
	$A$ en une entrée $I_B$ du problème $B$ et de décodage de toute sortie $O_B$ en une sortie $O_A$.
	\end{déf}

	\begin{axiome} Pour $\prec$, une relation d'ordre sur la priorité des opérateurs logiques. On prend~:
	\[\Leftrightarrow \quad\prec\quad \Rightarrow \quad\prec\quad \lor \quad\prec\quad \land \quad\prec\quad \lnot.\]
	\end{axiome}

	Afin d'étudier une formule logique, on peut construire son \textit{arbre de lecture} en séparant la formule aux opérateurs, par ordre
	croissant de priorité.

	\begin{déf} Pour $P$, un ensemble de propositions, on appelle \textit{fonction d'interprétation} (ou \textit{valuation}) toute fonction~:
	\[V : P \to \{0, 1\} : p \mapsto V(p).\]
	\end{déf}

	\begin{déf} Soit $\phi$ une formule bien formée sur un ensemble de propositions $P$. On définit sa \textit{valeur de vérité} évaluée en $V \in \{0, 1\}^P$
	par la fonction~:
	\[\logeval \phi_\cdot : \{0, 1\}^P \to \{0, 1\} : V \mapsto \logeval \phi_V,\]
	dont la	valeur est induite syntaxiquement.

	Pour $V$ une valuation, lorsque $\logeval \phi_V = 1$, on note $V \models \phi$, que l'on lit \textit{$V$ satisfait $\phi$}.
	\end{déf}

	\begin{déf} Soit $\phi$, une formule sur $P$.
	\begin{itemize}
		\item lorsque $\logeval \phi_\cdot^{-1}(\{1\}) \neq \emptyset$, on dit que $\phi$ est \textit{satisfaisable}~;
		\item lorsque $\logeval \phi_\cdot^{-1}(\{1\}) = \{0, 1\}^P$, on dit que $\phi$ est \textit{valide}.
	\end{itemize}
	\end{déf}

	\begin{lem} Soit $\phi$, une formule sur $P$. Si $\phi$ est valide, alors $\phi$ est satisfaisable.
	\end{lem}

	\begin{déf} Soient $n \in \N^*$, $\{\phi_i\}_{i \in \intint 1n}$ et $\phi$, des formules sur $P$. On dit que $\phi$ est une conséquence logique de
	$\{\phi_i\}_{i \in \intint 1n}$ lorsque la formule~:
	\[\bigwedge_{i=1}^n\phi_i \Rightarrow \phi\]
	est valide. On note cela $\phi_1, \ldots, \phi_n \models \phi$.
	\end{déf}

	\begin{déf} Deux formules $\phi$ et $\psi$ sur $P$ sont dites \textit{équivalentes} lorsque $\phi \Leftrightarrow \psi$ est valide. On note cela
	$\phi \equiv \psi$.
	\end{déf}

	\begin{thm} Une formule $\phi$ sur $P$ est valide si et seulement si sa négation est non-satisfaisable.
	\end{thm}

	\begin{proof} Soit $\phi$, une formule valide sur $P$. On sait alors que $\logeval \phi_\cdot^{-1}(\{1\}) = \{0, 1\}^P$. On en déduit que~:
	\[\logeval\phi_\cdot^{-1}(\{0\}) = \{0, 1\}^P \setminus \logeval \phi_\cdot^{-1}(\{1\}) = \emptyset.\]

	Or $\forall V \in \{0, 1\}^P : \logeval \phi_V = 1-\logeval {\lnot\phi}_V$. On en déduit que~:
	\[\logeval {\lnot \phi}_\cdot^{-1}(\{1\}) = \logeval {\phi}_\cdot^{-1}(\{0\}) = \emptyset.\]

	Idem pour $\Leftarrow$
	\end{proof}

	\begin{rmq} On en déduit qu'un algorithme qui détermine la satisfaisabilité d'une expression permet également de déterminer la validité.
	\end{rmq}

	\begin{déf} Un \textit{littéral} est soit une proposition $x \in P$, soit la négation $\lnot x$ d'une proposition.
	\end{déf}

	\begin{déf} Un ensemble $S$ de littéraux est dit \textit{satisfaisable} lorsqu'il ne contient pas une paire de littéraux complémentaires, i.e.~:
	\[\forall x \in S : \lnot x \not \in S.\]
	\end{déf}

	Pour déterminer la satisfaisabilité d'une formule, on peut créer son arbre sémantique par application des $\land$-règles et $\lor$-règles. Pour cela,
	on part de de la formule, et on applique le pas de simplification soit dur une conjonction, soit sur une disjonction (en ayant transformé tous les
	autres opérateurs en conjonctions/disjonctions au préalable).

	\begin{ex}
	\begin{align*}
		\phi \coloneqq &(x \lor y) \land (\lnot x \land \lnot y) \\
		&\left\{(x \lor y) \land (\lnot x \land \lnot y)\right\} \\
		&\left\{(x \lor y, \lnot x \land y\right\} \\
		&\left\{x \lor y, \lnot x, \lnot y\right\} \\
		&\left\{x, \lnot x, \lnot y\right\} \quad  \left\{y, \lnot x, \lnot y\right\}.
	\end{align*}

	Tous les sous-ensembles de littéraux sont non-satisfaisables, donc la formule $\phi$ est non-satisfaisable (i.e. $\lnot \phi$ est valide).
	\end{ex}

	L'algorithme de création de tableau sémantique pour SAT est le suivant. Soit $\phi$ une formule sur $P$. On construit l'arbre $T_\phi$ comme suit~:
	\begin{enumerate}
		\item Initialisation~: l'arbre est défini par $T_\phi = \{\phi\}$.
		\item Tant qu'il existe une feuille $\mathcal L \in T_\phi$ telle que $\exists \psi \in \mathcal L$, une formule simplifiable (donc contenant une
		conjonction ou une disjonction)~:
		\begin{itemize}
			\item si $\psi$ est simplifiable par une $\land$-règle, on ajoute une feuille $\mathcal L'$ à $\mathcal L$ telle que~:
			\[\mathcal L' = \left(\mathcal L \setminus \{\psi\}\right) \cup \{\psi_1, \psi_2\},\]
			pour $\psi_1$ et $\psi_2$, les deux sous-formules simplifiées de $\psi$~;
			\item si $\psi$ est simplifiable par une $\lor$-règle, on ajoute deux feuilles $\mathcal L_1$ et $\mathcal L_2$ à $\mathcal L$
			telles que~:
			\[\forall i \in \intint 12 : \mathcal L_i = \left(\mathcal L \setminus \{\psi\}\right) \cup \{\psi_i\},\]
			avec $\psi_1$ et $\psi_2$, les deux sous-formules simplifiées de $\psi$.i
		\end{itemize}
		\item S'il existe $\mathcal L \in T_\phi$, une feuille telle que $\mathcal L$ est satisfaisable, alors retourner \texttt{SATISFAISABLE}, sinon retourner
		\texttt{NON-SATISFAISABLE}.
	\end{enumerate}

\section{Déduction naturelle}
	\begin{déf} Soient $n \in \N^*$, $\{\phi_i\}_{i \in \intint 1n}$, $\psi$ des formules sur $P$. Si $\psi$ peut être dérivé des $\{\phi_i\}_{i \in \intint 1n}$,
	on appelle ces derniers des \textit{prémisses} et $\psi$ la \textit{conclusion}. On note cela~:
	\[\phi_1, \ldots, \phi_n \vdash \psi.\]

	On appelle cela un \textit{séquent}.
	\end{déf}

	La déduction naturelle fonctionne par élimination et introduction successives d'opérateurs.

	\begin{itemize}
		\item $\frac {\phi \quad \psi}{\phi \land \psi}\land_i$ se lit \textit{si $\phi$ et si $\psi$, alors $\phi$ et $\psi$}~;
		\item $\frac {\phi \land \psi}{\psi}\land_{e_1}$ se lit \textit{si $\phi$ et $\psi$, alors en particulier $\psi$}.
	\end{itemize}

	De même pour $\land_{e_2}$, et $\lor_i$. La double négation fonctionne également par introduction et élimination.

	La règle d'élimination de l'implication s'appelle \textit{Modus Ponens} (MP) et la règle d'élimination de l'implication par contraposée s'appelle
	\textit{Modus Tollens} (MT)~:
	\begin{itemize}
		\item $\frac {\phi \quad \phi \Rightarrow \psi}{\psi}MP$ se lit \textit{si $\phi$ et si $\phi$ implique $\psi$, alors $\psi$}~;
		\item $\frac {\lnot \psi \quad \phi \Rightarrow \psi}{\lnot \phi}MT$ se lit \textit{si $\phi$ implique $\psi$ et non-$\psi$, alors non-$\phi$}.
	\end{itemize}

	Afin d'introduire l'implication, on se sert du MT~:

	\begin{tabular}{l l l}
	1. & $x \Rightarrow y$ & prémisse \\
	2. & $\lnot y$ & hyp. \\
	3. & $\lnot x$ & MT 1, 2~; fin hyp. 2 \\
	4. & $\lnot \Rightarrow \lnot y$ & $\Rightarrow_i$ 2, 3
	\end{tabular}
	\qquad On déduit donc $\lnot y \Rightarrow \lnot x$.

	De manière plus générale, si en faisant l'hypothèse $\phi$, on arrive à la conclusion $\psi$, pour $\phi, \psi$ deux formules sur $P$, alors~:
	\[\frac {\begin{tabular}{l l}$\phi$ & \text{hyp.}\\\vdots & \\$\psi$ & \text{fin hyp.}\end{tabular}}{\phi \Rightarrow \psi}\Rightarrow_i\]

	\begin{rmq} Les prémisses et les hypothèses sont fondamentalement différentes~! Une hypothèse peut être émise même sans hypothèse, e.g.~:

	\begin{tabular}{l l l}
	1. & $p$ & hyp. \\
	2. & $\lnot\lnot p$ & $\lnot\lnot_i$ 1~; fin hyp. 1 \\
	3. & $p \Rightarrow \lnot\lnot p$ & $\Rightarrow_i$ 1, 2
	\end{tabular}

	On peut donc déduire de cela que $\vdash p \Rightarrow \lnot\lnot p$.
	\end{rmq}

	Afin d'éliminer la disjonction, on procède de la sorte~:

	\[\frac {
		\begin{tabular}{l l l l l}
			& $\phi_1$ & hyp. & $\phi_2$ & hyp. \\
			& \vdots   &       &  \vdots  & \\
			$\phi_1 \lor \phi_2$ & $\psi$ & fin hyp. & $\psi$ & fin hyp.
		\end{tabular}
	}{\psi}\lor_e\]

	\begin{déf} Toute formule $\phi$ sur $P$ telle que $\vdash \phi$ est appelée \textit{théorème}.
	\end{déf}

	Un théorème n'a donc pas besoin de prémisse. De plus, $\vdash \phi$ si et seulement si $\models \phi$, donc les théorèmes coïncident avec les formules valides.

	\begin{lem} Soient $n \in \N^*$, $\{\phi_i\}_{i \in \intint 1n}$, $\psi$ des formules sur $P$. Alors~:
	\[\phi_1, \ldots, \phi_n \vdash \psi \qquad \text{ si et seulement si }
		\qquad \vdash \phi_1 \Rightarrow \phi_2 \Rightarrow \ldots \Rightarrow \phi_n \Rightarrow \psi.\]
	\end{lem}

	Afin d'introduire la négation, on suppose une formule, et on en dérive $\bot$, ce qui permet d'en déduire sa négation. Cela se formule~:
	\[\frac {\begin{tabular}{l l}$\phi$ & hyp. \\ \vdots &\\$\bot$ & fin hyp.\end{tabular}}{\lnot\phi}\lnot_i\]

	Un raisonnement par l'absurde (reductio ad absurdum) se formalise par une introduction de double négation~:

	\[\frac {\begin{tabular}{l l}$\lnot \phi$ & hyp. \\ \vdots &\\ $\bot$ & fin hyp.\end{tabular}}{\lnot\lnot \phi}RAA,\]
	qui se note~:
	\[\frac {\phi}{\lnot\lnot\phi}\lnot\lnot_i.\]

	Ce qui est également équivalent à la loi du tiers exclus, que l'on peut formuler $\phi \Rightarrow \lnot(\lnot\phi)$, ou encore~:
	\[\frac {}{\phi \lor \lnot \phi}LEM.\]

	\begin{thm}Soient $n \in \N^*$, $\{\phi_i\}_{i \in \intint 1n}$, $\psi$ des formules sur $P$. Alors~:
	\begin{itemize}
		\item[(Adéquation)] $\left(\phi_1, \ldots, \phi_n \vdash  \psi\right) \Rightarrow \left(\phi_1, \ldots, \phi_n \models \psi\right)~;$
		\item[(Complétude)] $\left(\phi_1, \ldots, \phi_n \models \psi\right) \Rightarrow \left(\phi_1, \ldots, \phi_n \vdash  \psi\right).$
	\end{itemize}
	\end{thm}

	\begin{déf} Une formule $\phi$ sur $P$ est dit en \textit{forme normale conjonctive} (FNC ou CNF) lorsqu'elle s'exprime comme suit~:
	\[\phi \equiv \bigwedge_{i=1}^n\left(\bigvee_{j=1}^m\ell_{ij}\right),\]
	et est dite en \textit{forme normale disjonctive} (FND ou DNF) lorsqu'elle s'exprime comme suit~:
	\[\phi \equiv \bigvee_{i=1}^n\left(\bigwedge_{j=1}^m\ell_{ij}\right),\]
	avec $\ell_{ij}$ des littéraux sur $P$.
	\end{déf}
\end{document}
